\subsection{O Problema do Clique}

Um clique em um grafo G não orientado é um subconjunto C de vértices tal que, para cada $v$ e $w$ em C com $v \neq w$, ($v$,$w$) é uma aresta.
Para um grafo não orientado G e um inteiro $k$ como entrada,decidir se há um clique em G de tamanho mínimo $k$ \cite{goodrichprojeto}.

O problema do clique é NP. Pois podemos verificar se existe no mínimo $k$ vértices em C e também se há uma aresta para cada par de vértices em C, em tempo polinomial. Para provar que o problema do clique é NP-Completo basta demonstrar que SAT pode ser reduzido ao problema do clique.
