\section{Problemas com tempo polinomial}

Problemas com tempo polinomial ou apenas problemas \textbf{P}.
São problemas comprovadamente possíveis de serem resolvidos deterministicamente em tempo polinomial, isto é, esses problemas podem ser resolvidos no tempo $O(n^k)$ para alguma constante $k$ e $n$ o tamanho da entrada \cite{leisersonalgoritmos}.

Um algoritmo só pode ser considerado polinomial se e somente se o algoritmo obter tempo polinomial em uma Máquina de Turing \cite{HOPCROFT1974}.

Considerando apenas os problemas de decisão, ou seja, problemas que tem como resposta \textit{"sim"} e \textit{"não"}, na classe de problemas \textbf{P} temos por exemplo, o problema de verificar se determinada chave existe ou não em um vetor ordenado, esse problema pode ser resolvido em $O(lg n)$ \cite{NEAPOLITAN1997}.

A quantidade de problemas de decisão em \textbf{P} é imensa, porém podemos afirmar que nem todos os problemas de decisão são problemas \textbf{P}, pois existem alguns problemas que não são polinomiais como por exemplo a \textit{Aritmética de Presburger} que tem o pior tempo em $O(2^{2cn})$ \cite{NEAPOLITAN1997}.

