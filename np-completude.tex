\documentclass[12pt,a4papper]{article}


\usepackage[utf8x]{inputenc}
\usepackage[brazil]{babel}
%\usepackage{setspace} %pacote para espaçamento
\usepackage{indentfirst}
\usepackage{graphicx, wrapfig, float} %pacotes figuras/imagens
\usepackage{amsfonts, amssymb, amsmath, mathrsfs} %pacotes matémáticos
%\usepackage[top=2.5cm, bottom=3cm, left=2.5cm, right=2.5cm]{geometry} %margem
%\usepackage{algorithmic, algorithm} %algoritmos

%\onehalfspace %espacamento de um e meio
%\doublespace %espaçamento duplo

\title{NP-Completude}

\author{Breno Naodi Kusunoki \\
        \texttt{brenokusunoki@gmail.com}
        \and 
        Luiz Guilherme Castilho Martins \\
        \texttt{luizgui@gmail.com}
} 

\date{Londrina, \today}

 

\date{Londrina, \today}

\begin{document}

\maketitle
\thispagestyle{empty}
\newpage

\tableofcontents
\thispagestyle{empty}
\newpage

\section{Problemas com tempo polinomial}

Problemas com tempo polinomial ou apenas problemas \textbf{P}.
São problemas comprovadamente possíveis de serem resolvidos deterministicamente em tempo polinomial, isto é, esses problemas podem ser resolvidos no tempo $O(n^k)$ para alguma constante $k$ e $n$ o tamanho da entrada \cite{leisersonalgoritmos}.

\section{Problemas polinomiais não determinísticos}
Problemas polinomiais não determinísticos ou apenas problemas \textbf{NP}. Está é uma classe de problemas que não se conseguiu até esse momento criar algorítmos determinísticos que os resolvam em tempo satisfatório.

São problemas que não se tem um algorítmo determinístico com tempo polinomial que os resolvem, mas que podem ser verificados ou certificados em tempo polinomial. Verificar significa que dado uma possível solução para um problema \textbf{NP} é possível verificar se essa solução está correta em tempo polinomial ou seja $O(n^k)$.

Os problemas \textbf{P} também são considerados problemas \textbf{NP}, uma vez que podemos resolve-los 

\section{Problemas NP-Difícil}
Na classe de problemas \textbf{NP-difícil} estão os problemas considerados tão difíceis quanto qualquer outro problema \textbf{NP}. Dizer que um problema é tão difícil quando outro é um tanto subjetivo, mas por definição um problema é tão difícil quanto outro se for possível realizar uma redução polinomial entre eles \cite{HOPCROFT1974}.

Os problemas \textbf{NP-difíceis} não são necessariamente problemas \textbf{NP}, existe uma intersessão entre as duas classes como pode ser visto na IMAGE .

\section{Problemas NP-Completo}

A classe \textbf{NP-completo} é uma classe de problemas onde todos eles estão em \textbf{NP} e também podem ser reduzidos polinomialmente aos problemas \textbf{NP}. 
O \textit{Problema da Satisfatibilidade} ou apenas \textit{SAT} foi o primeiro entre os \textbf{NP-completo} através do \textit{Teorema de Cook} onde foi demonstrado matematicamente que o problema \textbf{SAT} era polinomialmente redutível a todos os problemas \textbf{NP} e também estava em \textbf{NP}.

Após Cook apresentar o \textbf{Teorema de Cook} e demostrar o primeiro \textbf{NP-completo} começou a surgir inúmeros outros, isso ocorreu através da transitividade existente na redução polinomial. Informalmente podemos dizer que se existe uma redução de \textbf{A} a \textbf{B} e também existe uma redução de \textbf{B} a \textbf{C} então pela transitividade de reduções,  podemos dizer que existe a redução de \textbf{A} a \textbf{C}. 

\section{Redutibilidade}
Redutibilidade é utilizado na prova de NP-Completude. Reduzir um problema significa dizer que dado dois problemas \textbf{A} a \textbf{B}, existe uma função com tempo polinomial que transforma qualquer instância de \textbf{A} em instância de \textbf{B}, existindo também a condicional \textit{se somente se} sobre a resposta, ou seja, se a resposta de \textbf{A} é "sim" a de \textbf{B} também é "sim".


\section{Transferência de cotas}
Transferência de cotas se dá quando há uma redução de um problema em outro, dessa forma existe uma transferência de cota inerente da redução.

Dada uma redução polinomial de \textbf{X} a \textbf{Y} ou seja $X \leq_P Y$ e o algoritmo de redução $f$ de tempo polinomial. Sabendo que a cota superior de \textbf{Y} é $O\,(nlgn)$ existirá uma transferência da cota superior de \textbf{Y} para a cota inferior de \textbf{X} acrescido do tempo do algoritmo de redução ou seja o a cota inferior de \textbf{X} será $O\,(nlgn\,+\,f)$


\section{Exemplo}

\subsection{O Problema do Clique}

Um clique em um grafo G não orientado é um subconjunto C de vértices tal que, para cada $v$ e $w$ em C com $v \neq w$, ($v$,$w$) é uma aresta.
Para um grafo não orientado G e um inteiro $k$ como entrada,decidir se há um clique em G de tamanho mínimo $k$ \cite{goodrichprojeto}.

O problema do clique é NP. Pois podemos verificar se existe no mínimo $k$ vértices em C e também se há uma aresta para cada par de vértices em C, em tempo polinomial. Para provar que o problema do clique é NP-Completo basta demonstrar que SAT pode ser reduzido ao problema do clique.



\nocite{SKIENA2010}

\bibliographystyle{ieeetr}
\bibliography{bibliografia}


\end{document}
